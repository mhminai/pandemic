\section{Introduction}\label{introduction}

Employee engagement is getting increasing traction due to the dynamic
nature of the environment faced by organizations nowadays, Shuck (2011)
opines that ``\ldots{} engaging employees no matter the industry has
become a strategic imperative \ldots{}''(p.317). Bakker and Leiter
(2010) and stress that the full capabilities of employees need to be
brought to bear if organizations are to compete in the current
environment. Its impact on financial performance at the business unit
level has been demonstrated, and validated in at least one diary study.

\section{Theory and Hypothesis}\label{theory-and-hypothesis}

\subsection{Leader-member exchange}\label{leader-member-exchange}

LMX theory builds upon social exchange theory (Blau 1964) to explain the
differential relationships which supervisors have with their
subordinates. Interaction episodes within the dyad are expected to give
rise to reciprocal exchanges that are equitable for both parties. A low
quality, transactional relationship is based on formal requirements of
the job contract and economic exchanges, whereas a high quality,
transformational relationship matures to include social exchanges and
intangible benefits for both parties. Some of the proposed intangible
benefits that are accessible to followers in high quality LMX
relationships are: mutual trust, respect, liking; support,
consideration; greater lattitude over duties and responsibilities and
more feedback and support for development.

\emph{H1: LMX would be positively related to work engagement, such that
employees perceiving a higher quality of LMX with their supervisors
would also report greater work engagement}

\subsubsection{Core self-evaluations}\label{core-self-evaluations}

People high on self-evaluations would positively appraise themselves
across situations by virtue of their feeling of capability
(self-efficacy), worthiness (self-esteem), composure (low on
neuroticism) and feeling of agency (internal locus of control). in a
contingency approach to leadership, experimentally show that state core
self-evaluations can act as a substitute for transformational
leadership. Graen and Uhl-Bien (1995) have claimed that high LMX quality
is behaviorally similar to transformational leadership. Therefore a
similar argument can be made for high LMX quality. In that employees low
in self-evaluations benefit more from a better quality of relationship
with their leaders than employees that already are high in their
self-evaluations. Further they conceptualize positive CSE's as
``\ldots{} an inner resource for feelings of competence and
self-determination which may lower the need for contextual support
\ldots{}''. Empirically also a component of core self-evaluations
(locus-of-control) has also been shown to be a moderator of the LMX -
performance relationship.

Therefore we hypothesize that:

\emph{H2a: Core self-evaluations would be positively related to work
engagement, such that employees reporting higher core self-evaluations
would also report higher work engagement}

\emph{H2b: Core self-evaluations would moderate the relationship between
perceived LMX quality and work engagement, such that the relationship
between perceived LMX quality and work engagement would be higher for
people lower in core self-evaluations}

\subsubsection{P-O Fit}\label{p-o-fit}

Keeping the previous reasons in mind we propose our final two
hypotheses:

\emph{H3a: Employee reported P-O Fit will be positively related to work
engagement, such that employees perceiving greater value congruence with
their organizations would also be more engaged.}

\emph{H3b: P-O Fit would moderate the relationship between quality of
LMX and work engagement, such that the relationship between LMX and work
engagement would be stronger for employees reporting a strong fit
between self and organizational values.}

\section{Method}\label{method}

\subsection{Procedure and
Participants}\label{procedure-and-participants}

\textbf{Need to talk about the sample characteristics and see if this
matches with the ``average'' characteristics of IT services employees in
India}

An Internet based survey was used, however participants were not
recruited via the internet. The relevant precautions such as ensuring
only one response per IP address were implemented. As the design was
cross sectional in nature, we also implemented some of the procedural
remedies for common method biases suggested in P. M. Podsakoff et al.
(2003). Survey questions corresponding to each construct were placed on
different pages of a multi-page web survey. The participants were
assured of confidentiality and no identifying information was collected
to encourage openness in responses. The criterion variable was measured
before the predictor variables, psychological separation was attempted
by using different introductions for each page. The survey was initially
tested on a sample of 11 students to ensure that all items were
comprehensible and instructions were clear. The use of a web based
survey allowed us to incorporate help tips which were displayed below
the questions for those items that were found to have comprehension
issues in the testing phase. A total of 309 people attempted to respond
but 32 dropped off before completing the survey, thus a completion rate
of about 90\% was achieved.

\subsection{Measures}\label{measures}

\subsubsection{Work Engagement}\label{work-engagement}

Engagement was measured using the short version of the Utrecht Work
Engagement Scale. This is the ``\ldots{} most often used scientifically
derived measure of engagement \ldots{}'' (Bakker, Albrecht, and Leiter
2011, 9). The items were measured on a 7 point (0 - 6) scale. A sample
item from the scale is ``I am enthusiastic about my job'' (0 = Never; 6
= Always).

\subsubsection{Leader-member exchange}\label{leader-member-exchange-1}

Leader-member exchange quality was measured using the LMX 7 scale
recommended by Graen and Uhl-Bien (1995). This measure has been found to
have sound psychometric properties in relation to other measures of LMX,
with member reported LMX having marginally better reliabilities. also
report that LMX 7 and its multi-dimensional variant are alternate forms
of the same instrument. The items were scored on a 5 point scale. An
example item being ``How well does your leader recognize your
potential?'' (1 = Not at all; 5 = Fully).

\subsubsection{Core self-evaluations}\label{core-self-evaluations-1}

Core self-evaluations were measured with the core self-evaluations
scale. This is the only scale to measure the gestalt self-evaluations
proposed by, and has been used in other studies that hypothesize about
core self-evaluations. This is a balanced 12-item scale with equal
positively and negatively scored items. An example item is ``Overall, I
am satisfied with myself''. It is a likert-type scored from 1 - 7, with
1 being ``Strongly disagree'' and 7 being ``Strongly agree''.

\subsubsection{P-O Fit}\label{p-o-fit-1}

Value congruence with organization was operationalized with the three
item P-O Fit scale from. A 7 point scale was used for this measure. An
example item is ``I feel the values and personality of this organization
reflect my own values and personality'' (1 = Stongly disagree; 7 =
Strongly agree).

\subsubsection{Control Variables}\label{control-variables}

To control for effects that were not substantive from the perspective of
our study we included five control variables. Bakker, Tims, and Derks
(2012) has speculated that people with higher education are likely to be
more engaged, thus we measured education using a categorical variable
(coded as: 1-12th class or less, 2-Diploma, 3-Bachelor's degree,
4-Post-graduate degree). Tenure has been found to be significant in
studies of LMX. Therefore we measured tenure in 5 bins (1-Less than a
year, 2-1-2 years, 3-3-5 years, 4-6-10 years and 5-More than 10 years).
Age has also been found to correlate with work engagement thus we
controlled for age as well. This was measured at intervals of 5 years
from 25 to 60, with first and last bins being ``Less than 25'' and
``More than 60''. However we did not receive any response from people
above 40, thus effectively we only had 4 bins (1-``less than 25'',
2-``26 to 30'', 3-``31 to 35'' and 4-``35 to 40''). We also controlled
for gender (0-Male and 1-Female), this too has been found to correlate
with some dimensions of engagement. Finally we also controlled for the
role of the respondent (1-Individual contributor and 2-Manager), as
having subordinates might change one's perception of LMX quality with a
superior.

\section{Analysis and Results}\label{analysis-and-results}

Internal consistency of the scales used was first assessed using
Cronbach's alpha. This analysis showed that the reliability of LMX and
work engagement would increase if we dropped one item each from these
scales. Thus one item from LMX scale (``Do you know where you stand with
your supervisor {[}and{]} do you usually know how satisfied your
supervisor is with what you do?'') and one item from UWES (" I am
immersed in my job``) were dropped. Though the reliabilities were been
affected only by 0.01, the power of interaction tests drops when scales
lack reliability, therefore we favored dropping these items. Table 1
shows the means, standard deviations and zero order correlations among
the variables. The diagonal has the alphas for each of the scales used.
The reported internal consistencies (α) are without the dropped items.

\begin{verbatim}
 --------------------------------------------
            Table 1 about here               
 --------------------------------------------
\end{verbatim}

The interaction plots of LMX with core self-evaluations and P-O Fit are
shown in Figure 1 and Figure 2 respectively. We have plotted the
relationship between LMX and work engagement at the mean and one
standard deviation above and below the mean for the moderators. From
Figure 1 we can see that the slope of the association between LMX and
work engagement is attenuated as the respondents core self-evaluations
increase. Thus LMX is negatively moderating the effect of LMX on work
engagement, providing support for Hypothesis 2b. In Figure 2 we can see
that for low values of value congruence with organization, LMX has
essentially no effect on work engagement, however at higher values of
P-O Fit LMX has significant effect on work engagement. Implying that P-O
Fit conceptualized as value congruence with organizational values
catalyzes the effect of LMX quality on work engagement, providing
support for Hypothesis 3b.

\begin{verbatim}
 --------------------------------------------
          Insert Figure 2 about here
 --------------------------------------------
\end{verbatim}

\section{Discussion}\label{discussion}

In this study we empirically tested the role of core self-evaluations
and person-organization value congruence in the relationship between
perceived LMX quality and work engagement. This contextual approach to
the study of LMX is important as it helps explain inconsistent
relationships that LMX has with its outcomes. In particular suggests
that definitional drift in LMX as noted by could be due to the lack of
exploration of contextual influences.

\subsection{Limitations and Future
work}\label{limitations-and-future-work}

There are some distinct limitations with our study and these deserve to
be discussed. The first major limitation is that the cross-sectional
nature of our study excludes and causal inferences. In fact the JD-R
model itself suggests the reciprocal relationships between engagement
and resources (Bakker, Tims, and Derks 2012). found that activated forms
of well-being (engagement) could lead to worse want-actual fits, with
greater motivation increasing the wanted level of job features.

\section*{References}\label{references}
\addcontentsline{toc}{section}{References}

Bakker, Arnold B., Simon L Albrecht, and Michael P Leiter. 2011. ``Key
Questions Regarding Work Engagement.'' \emph{European Journal of Work
and Organizational Psychology} 20 (1): 4--28.
doi:\href{http://dx.doi.org/10.1080/1359432X.2010.485352}{10.1080/1359432X.2010.485352}.

Bakker, Arnold B., and Michael P. Leiter, eds. 2010. \emph{Work
Engagement: A Handbook of Essential Theory and Research}. 1 edition.
Hove England ; New York: Psychology Press.

Bakker, Arnold B., M. Tims, and D. Derks. 2012. ``Proactive Personality
and Job Performance: The Role of Job Crafting and Work Engagement.''
\emph{Human Relations} 65 (10): 1359--78.
doi:\href{http://dx.doi.org/10.1177/0018726712453471}{10.1177/0018726712453471}.

Blau, Peter Michael. 1964. \emph{Exchange and Power in Social Life}.
Transaction Publishers.

Graen, George B, and Mary Uhl-Bien. 1995. ``Relationship-Based Approach
to Leadership: Development of Leader-Member Exchange (LMX) Theory of
Leadership over 25 Years: Applying a Multi-Level Multi-Domain
Perspective.'' \emph{The Leadership Quarterly} 6 (2): 219--47.

Podsakoff, Philip M., Scott B. MacKenzie, Jeong-Yeon Lee, and Nathan P.
Podsakoff. 2003. ``Common Method Biases in Behavioral Research: A
Critical Review of the Literature and Recommended Remedies.''
\emph{Journal of Applied Psychology} 88 (5): 879--903.
doi:\href{http://dx.doi.org/10.1037/0021-9010.88.5.879}{10.1037/0021-9010.88.5.879}.

Shuck, Michael Bradley. 2011. ``Integrative Literature Review: Four
Emerging Perspectives of Employee Engagement: An Integrative Literature
Review.'' \emph{Human Resource Development Review} 10 (3): 304--28.
doi:\href{http://dx.doi.org/10.1177/1534484311410840}{10.1177/1534484311410840}.
